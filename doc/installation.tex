

\Chapter{Installation}

Installation of the package is fairly easy.
Fetch the latest qaos package at

\url{http://www.math.tu-berlin.de/~kant/download/gap/qaos.tar.bz2}

or via FTP at

\url{ftp://ftp.math.tu-berlin.de/pub/algebra/Kant/contrib/gap/}




\Section{Installation of the GAP package}

If you have permission to add files to the installation of GAP 4 on your system
you may install the qaos package into the \file{pkg/} subdirectory of the
GAP installation tree.

\beginexample
shell> cd /path/to/GAP4/installation/tree/
shell> cd pkg/
shell> tar xjf /path/to/qaos.tar.bz2
\endexample

This yields another subdirectory called \file{qaos/} with all the necessary
files.

If you do not have the permission to install the package globally just install
it to some private area, for example your home directory.

\beginexample
shell> cd ~
shell> mkdir mygap
shell> mkdir mygap/pkg
shell> cd mygap/pkg/
shell> tar xjf /path/to/qaos.tar.bz2
\endexample

Now whenever you start GAP, be sure to pass the \file{mygap/} directory to the
package search path of GAP.

\beginexample
shell> gap -l ";$HOME/mygap"
\endexample




\Section{Installation of cURL}

Go to \url{http://curl.haxx.se} and fetch the latest release of cURL for your
system.  Install it.  Refer to cURL installation instructions if necessary.

If you have downloaded precompiled binary packages for your system and none of
them seem to work, you may also try installing cURL via sources.
Just fetch the source archive, unpack it somewhere and say

\beginexample
shell> ./configure && make && make install
\endexample

Finally, you can test for a successful curl installation by

\beginexample
shell> curl http://curl.haxx.se
\endexample

If this command spits out lots of HTML into your terminal everything is
installed properly.  If not, adjust your \var{$PATH} variable such that 

\beginexample
shell> which curl
\endexample

finds a valid path to the curl binary.





